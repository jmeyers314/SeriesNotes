\documentclass{article}
\usepackage[utf8]{inputenc}
\usepackage{geometry}
\usepackage{color}
\usepackage{graphicx}
\usepackage{amsmath}

\geometry{textwidth=6.5in, textheight=9.0in,
  marginparsep=7pt, marginparwidth=.6in}
\setlength{\parindent}{0in}
\setlength{\parskip}{0.08in}

\newcommand{\red}[1]{\textcolor{red}{#1}}
\newcommand{\blue}[1]{\textcolor{blue}{#1}}
\newcommand{\farcs}{\mbox{\ensuremath{.\!\!^{\prime\prime}}}}

\title{Spergel Notes}
\author{Josh Meyers}
\date{December 2014}

\begin{document}

I want to expand:

\begin{equation}
  \Sigma(\vec{k}|L,\gamma,r_0,\epsilon,\phi_0) = \frac{L}{\{1+k^2\gamma^2r_0^2[1-\epsilon\cos(2(\phi-\phi_0))]\}^{1+\nu}}.
\end{equation}
Instead of fitting for $r_0$, Spergel suggests fitting for
\begin{equation}
  \Delta = 1 - \left(\frac{r_0}{r_1}\right)^2,
\end{equation}
where $r_1$ is the size of the nearest precomputed profile.
Expand:
\begin{equation}
  \begin{split}
    \Sigma(\vec{k}) & = \frac{L}{\{1+k^2\gamma^2r_1^2(1-\Delta)[1-\epsilon\cos(2(\phi-\phi_0))]\}^{1+\nu}} \\
    & = \frac{L}{\{1+k^2\gamma^2r_1^2 - k^2\gamma^2r_1^2[\Delta + (1-\Delta)\epsilon\cos(2(\phi-\phi_0))]\}^{1+\nu}} \\
    & = \frac{L (1+k^2\gamma^2r_1^2)^{-1-\nu}}{\{1 - \frac{k^2\gamma^2r_1^2}{1+k^2\gamma^2r_1^2}\left[\Delta + (1-\Delta)\epsilon\cos(2(\phi-\phi_0))\right]\}^{1+\nu}} \\
    & = \frac{L}{(1+k^2\gamma^2r_1^2)^{1+\nu}}\sum_{j=0}^{\infty}\binom{\nu+j}{j}\left(\frac{k^2\gamma^2r_1^2}{1+k^2\gamma^2r_1^2}\right)^{j}[\Delta + (1-\Delta)\epsilon\cos(2(\phi-\phi_0))]^j \\
    & = \sum_{j=0}^{\infty}\frac{L(k\gamma r_1)^{2j}}{(1+k^2\gamma^2r_1^2)^{1+\nu+j}}\binom{\nu+j}{j}[\Delta + (1-\Delta)\epsilon\cos(2(\phi-\phi_0))]^j \\
    & = \sum_{j=0}^{\infty}\sum_{m=0}^{j}\frac{L(k\gamma r_1)^{2j}}{(1+k^2\gamma^2r_1^2)^{1+\nu+j}}\binom{\nu+j}{j}\binom{j}{m}\Delta^{j-m}(1-\Delta)^m\epsilon^m\cos^m(2(\phi-\phi_0))
  \end{split}
\end{equation}

This is where the mistake is.  We need to separate the $\phi$ and the $\phi_0$ in order for the galaxy model parameters to separate from the $\vec{k}$ basis functions.
\begin{equation}
  \begin{split}
    \cos^m(x) & = \frac{1}{2^m}\left(e^{ix} + e^{-ix}\right)^m \\
    & = \frac{1}{2^m} \sum_{n=0}^{m}\binom{m}{n}(e^{i n x} e^{-i (m-n) x}) \\
    & = \frac{1}{2^m} \sum_{n=0}^{m}\binom{m}{n}e^{i (2n-m) x}
  \end{split}
\end{equation}
And thus:
\begin{equation}
  \begin{split}
    \label{eqn:sum}
    \Sigma(\vec{k}) & = \sum_{j=0}^{\infty}\sum_{m=0}^{j}\sum_{n=0}^{m}\frac{L(k\gamma r_1)^{2j}}{2^m(1+k^2\gamma^2r_1^2)^{1+\nu+j}}\binom{\nu+j}{j}\binom{j}{m}\binom{m}{n}\Delta^{j-m}(1-\Delta)^m\epsilon^m\exp(i 2 (2n-m) (\phi-\phi_0)) \\
    & = \sum_{j=0}^{\infty}\sum_{m=0}^{j}\sum_{n=0}^{m} a_{jmn}\mu_{jmn}(\vec{k}),
  \end{split}
\end{equation}
where
\begin{equation}
  a_{jmn}(L, \Delta, \epsilon, \phi_0) = L \Delta^{j-m}(1-\Delta)^m\epsilon^m\exp(-i 2 (2n-m) \phi_0)
\end{equation}
and
\begin{equation}
  \mu_{jmn}(\vec{k}) = \frac{(k\gamma r_1)^{2j}}{2^m(1+k^2\gamma^2r_1^2)^{1+\nu+j}}\binom{\nu+j}{j}\binom{j}{m}\binom{m}{n}\exp(i 2 (2n-m) \phi).
\end{equation}

For galsim, let $k \gamma r_1 \rightarrow k$.
Note that
\begin{equation}
  a_{jmn} + a_{jm(m-n)} = L \Delta^{j-m}(1-\Delta)^m\epsilon^m2\cos(2 (2n-m) \phi_0)
\end{equation}
and
\begin{equation}
  \begin{split}
    \mu_{jmn} + \mu_{jm(m-n)} & = \frac{k^{2j}}{2^m(1+k^2)^{1+\nu+j}}\binom{\nu+j}{j}\binom{j}{m}\binom{m}{n}2 \cos(2 (2n-m) \phi) \\
    & = \frac{k^{2j}}{2^{m-1}(1+k^2)^{1+\nu+j}}
    \frac{\Gamma(\nu+j+1)}{\Gamma(\nu+1) j!}
    \frac{j!}{(j-m)!m!}
    \frac{m!}{(m-n)!n!}
    \cos(2 (2n-m) \phi) \\
    & = \frac{k^{2j}}{2^{m-1}(1+k^2)^{1+\nu+j}}
    \frac{\Gamma(\nu+j+1)}{\Gamma(\nu+1)}
    \frac{1}{(j-m)!}
    \frac{1}{(m-n)!n!}
    \cos(2 (2n-m) \phi)
  \end{split}
\end{equation}
We can further rearrange terms such that $\mu$ only depends on the combination $2n-m$:
\begin{equation}
  a_{jmn} + a_{jm(m-n)} = \frac{L \Delta^{j-m}(1-\Delta)^m\epsilon^m\cos(2 (2n-m) \phi_0)}
  {2^{m-2}(j-m)!(m-n)!n!}
\end{equation}
and
\begin{equation}
  \mu_{jmn} + \mu_{jm(m-n)} = \frac{k^{2j}}{(1+k^2)^{1+\nu+j}}
    \frac{\Gamma(\nu+j+1)}{\Gamma(\nu+1)}
    \cos(2 (2n-m) \phi)
\end{equation}
If we let $q=2n-m$ and restrict $q \ge 0$ (which we can do since $\cos$ is an even function), then we can rewrite Eqn. \ref{eqn:sum} as a double sum instead of a triple sum.
\begin{equation}
  \Sigma(\vec{k}) = \sum_{j=0}^{\infty}\sum_{q=0}^{j} a_{jq}\mu_{jq}(\vec{k}).
\end{equation}
To determine which $m$ and $n$ correspond to a given $q$, note that $m+q=2n$ is even and that $n \le m \Rightarrow (m+q)/2 \le m \Rightarrow q \le m$ sets a lower limit for $m$ and $m \le j$ sets the upper limit.
With this in mind, we determine that :
\begin{equation}
  a_{jq} = \begin{cases} \sum_{\substack{m=q\\m+q\,\mathrm{even}}}^{j}\frac{L \Delta^{j-m}(1-\Delta)^m\epsilon^m\cos(2 q \phi_0)}
    {2^{m-2}(j-m)!(\frac{m-q}{2})!(\frac{m+q}{2})!} & q > 0,\\
    \frac{L \Delta^{j-m}(1-\Delta)^m\epsilon^m}
         {2^{m-4}(j-m)!(\frac{m-q}{2})!(\frac{m+q}{2})!} & q = 0
  \end{cases}
\end{equation}
and
\begin{equation}
  \mu_{jq} = \frac{k^{2j}}{(1+k^2)^{1+\nu+j}}\frac{\Gamma(\nu+j+1)}{\Gamma(\nu+1)}\cos(2 q \phi)
\end{equation}
\end{document}
