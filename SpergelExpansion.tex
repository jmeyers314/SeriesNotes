\documentclass{article}
\usepackage[utf8]{inputenc}
\usepackage{geometry}
\usepackage{color}
\usepackage{graphicx}
\usepackage{amsmath}

\geometry{textwidth=6.5in, textheight=9.0in,
  marginparsep=7pt, marginparwidth=.6in}
\setlength{\parindent}{0in}
\setlength{\parskip}{0.08in}

\newcommand{\red}[1]{\textcolor{red}{#1}}
\newcommand{\blue}[1]{\textcolor{blue}{#1}}

\let\Re\undefined
\DeclareMathOperator{\Re}{Re}
\let\Im\undefined
\DeclareMathOperator{\Im}{Im}

\title{Spergel Notes}
\author{Josh Meyers}
\date{December 2014}

\begin{document}

I want to expand:

\begin{equation}
  \label{eqn:SBdef}
  \Sigma(\vec{k}|L,r_0,\epsilon,\phi_0) = \frac{L}{\{1+k^2r_0^2[1+\epsilon\cos(2(\phi-\phi_0))]\}^{1+\nu}}.
\end{equation}
Instead of fitting for $r_0$, Spergel suggests fitting for
\begin{equation}
  \Delta = 1 - \left(\frac{r_0}{r_1}\right)^2,
\end{equation}
where $r_1$ is the size of the nearest precomputed profile.  Here it's convenient to take $k r_1 \rightarrow k$ since $k$ and $r_1$ always occur together.
Expand:
\begin{equation}
  \begin{split}
    \Sigma(\vec{k}) & = \frac{L}{\{1+k^2(1-\Delta)[1+\epsilon\cos(2(\phi-\phi_0))]\}^{1+\nu}} \\
    & = \frac{L}{\{1+k^2 - k^2[\Delta + (\Delta-1)\epsilon\cos(2(\phi-\phi_0))]\}^{1+\nu}} \\
    & = \frac{L (1+k^2)^{-1-\nu}}{\{1 - \frac{k^2}{1+k^2}\left[\Delta + (\Delta-1)\epsilon\cos(2(\phi-\phi_0))\right]\}^{1+\nu}} \\
    & = \frac{L}{(1+k^2)^{1+\nu}}\sum_{j=0}^{\infty}\binom{\nu+j}{j}\left(\frac{k^2}{1+k^2}\right)^{j}[\Delta + (\Delta-1)\epsilon\cos(2(\phi-\phi_0))]^j \\
    & = \sum_{j=0}^{\infty}\frac{L k^{2j}}{(1+k^2)^{1+\nu+j}}\binom{\nu+j}{j}[\Delta + (\Delta-1)\epsilon\cos(2(\phi-\phi_0))]^j \\
    & = \sum_{j=0}^{\infty}\sum_{m=0}^{j}\frac{L k^{2j}}{(1+k^2)^{1+\nu+j}}\binom{\nu+j}{j}\binom{j}{m}\Delta^{j-m}(\Delta-1)^m\epsilon^m\cos^m(2(\phi-\phi_0))
  \end{split}
\end{equation}

This is where the mistake is.  We need to separate the $\phi$ and the $\phi_0$ in order for the galaxy model parameters to separate from the $\vec{k}$ basis functions.
\begin{equation}
  \begin{split}
    \cos^m(x) & = \frac{1}{2^m}\left(e^{ix} + e^{-ix}\right)^m \\
    & = \frac{1}{2^m} \sum_{n=0}^{m}\binom{m}{n}(e^{i n x} e^{-i (m-n) x}) \\
    & = \frac{1}{2^m} \sum_{n=0}^{m}\binom{m}{n}e^{i (2n-m) x}
  \end{split}
\end{equation}
And thus:
\begin{equation}
  \begin{split}
    \label{eqn:sum}
    \Sigma(\vec{k}) & = \sum_{j=0}^{\infty}\sum_{m=0}^{j}\sum_{n=0}^{m}\frac{L k^{2j}}{2^m(1+k^2)^{1+\nu+j}}\binom{\nu+j}{j}\binom{j}{m}\binom{m}{n}\Delta^{j-m}(\Delta-1)^m\epsilon^m\exp(i 2 (2n-m) (\phi-\phi_0)) \\
    & = \sum_{j=0}^{\infty}\sum_{m=0}^{j}\sum_{n=0}^{m} a_{jmn}\mu_{jmn}(\vec{k}),
  \end{split}
\end{equation}
where
\begin{equation}
  a_{jmn}(L, \Delta, \epsilon, \phi_0) = \frac{L}{2^m} \Delta^{j-m}(\Delta-1)^m\epsilon^m\binom{j}{m}\binom{m}{n}\exp(-i 2 (2n-m) \phi_0)
\end{equation}
and
\begin{equation}
  \mu_{jmn}(\vec{k}) = \frac{k^{2j}}{(1+k^2)^{1+\nu+j}}\binom{\nu+j}{j}\exp(i 2 (2n-m) \phi).
\end{equation}

We have arranged factors that could have occured in either $a_{jmn}$ or $\mu_{jmn}(\vec{k})$ in such a way that $\mu_{jmn}(\vec{k})$ only depends on $j$ and the combination $2n-m$ (and hence not on both $m$ and $n$ separately).
We take advantage of this by defining $q=2n-m$, which can take on values between $-j$ and $j$.
For given $j$ and $q$, the range of possible $m$ is $|q| < m < j$.
For given $q$ and $m$ the value of $n$ is uniquely determined as $n=(m+q)/2$.
This allows us to rewrite Eq. \ref{eqn:sum} as 
\begin{equation}
  \label{eqn:sum_jq}
  \Sigma(\vec{k}) = \sum_{j=0}^{\infty}\sum_{q=-j}^{j} a_{jq}\mu_{jq}(\vec{k}),
\end{equation}
where
\begin{equation}
  \label{eqn:ajq}
  a_{jq}(L, \Delta, \epsilon, \phi_0) = \sum_{\substack{m=|q|\\m+q\,\mathrm{even}}}^{j}\frac{L}{2^m} \Delta^{j-m}(\Delta-1)^m\epsilon^m\binom{j}{m}\binom{m}{\frac{m+q}{2}}\exp(-i 2 q \phi_0)
\end{equation}
and
\begin{equation}
  \label{eqn:mujq}
  \mu_{jq}(\vec{k}) = \frac{k^{2j}}{(1+k^2)^{1+\nu+j}}\binom{\nu+j}{j}\exp(i 2 q \phi).
\end{equation}

The coefficients $a_{jq}$ and basis functions $\mu_{jq}(\vec{k})$ are complex, but we know the surface brightness profile $\Sigma(\vec{k})$ is real (see Eq. \ref{eqn:SBdef}).
Inspecting Eqs. \ref{eqn:ajq} and \ref{eqn:mujq}, we see that
\begin{equation}
  a_{jq}\mu_{jq}(\vec{k}) = [a_{j(-q)}\mu_{j(-q)}(\vec{k})]^*,
\end{equation}
which implies
\begin{equation}
  \Re{[a_{jq}\mu_{jq}(\vec{k}) + a_{j(-q)}\mu_{j(-q)}(\vec{k})]} = 2\Re{[a_{jq}\mu_{jq}(\vec{k})]}
  = 2 (\Re{[a_{jq}]}\Re{[\mu_{jq}(\vec{k})]} - \Im{[a_{jq}]}\Im{[\mu_{jq}(\vec{k})]})
\end{equation}
and
\begin{equation}
  \Im{[a_{jq}\mu_{jq}(\vec{k}) + a_{j(-q)}\mu_{j(-q)}(\vec{k})]} = 0.
\end{equation}

We can use these relations to define $a^\prime_{jq}$ and $\mu^\prime_{jq}(\vec{k})$
\[
a^\prime_{jq} =
\begin{cases}
  2 \Re[a_{jq}], & q > 0, \\
  \Re[a_{jq}], & q = 0, \\
  -2 \Im[a_{jq}], & q < 0,
\end{cases}
\]
\[
\mu^\prime_{jq}(\vec{k}) =
\begin{cases}
  \Re[\mu_{jq}(\vec{k})], & q \ge 0, \\
  \Im[\mu_{jq}(\vec{k})], & q < 0,
\end{cases}
\]
such that
\begin{equation}
  \label{eqn:sum_jqprime}
  \Sigma(\vec{k}) = \sum_{j=0}^{\infty}\sum_{q=-j}^{j} a^\prime_{jq}\mu^\prime_{jq}(\vec{k}).
\end{equation}
Notice that we have been careful to not double count the terms $q$ and $-q$ when $q=0$.
This way all of the coefficients and basis functions are real, which is good because GalSim doesn't currently have the ability to handle complex-valued real-space images.

For a given series order $j_\mathrm{max}$, there are $(j_\mathrm{max}+1)^2$ terms, which is a factor $\sim 2$ more than Spergel(2010).
\end{document}
