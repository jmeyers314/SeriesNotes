\documentclass{article}
\usepackage[utf8]{inputenc}
\usepackage{geometry}
\usepackage[usenames,dvipsnames,svgnames,table]{xcolor}
%\usepackage{color}
\usepackage{graphicx}
\usepackage{amsmath}

\geometry{textwidth=6.5in, textheight=9.0in,
  marginparsep=7pt, marginparwidth=.6in}
\setlength{\parindent}{0in}
\setlength{\parskip}{0.08in}

\newcommand{\red}[1]{\textcolor{red}{#1}}
\newcommand{\blue}[1]{\textcolor{blue}{#1}}
\newcommand*{\annot}[1]{\tag*{\footnotesize{\textcolor{gray}{#1}}}}

\let\Re\undefined
\DeclareMathOperator{\Re}{Re}
\let\Im\undefined
\DeclareMathOperator{\Im}{Im}

\title{Moffat Notes}
\author{Josh Meyers}
\date{March 2015}

\begin{document}

We want to expand:

\begin{equation}
  \label{eqn:SBdef}
  \Sigma(\vec{r}|L,r_0,\epsilon,\phi_0) = \frac{L (\beta-1)\sqrt{1-\epsilon^2}}{\pi r_0^2 \{1+(r^2/r_0^2)[1-\epsilon\cos(2(\phi-\phi_0))]\}^\beta}.
\end{equation}

There are several modifications here from the Spergel profile expansion.
I switched the sign of $\epsilon$ since the fat direction in Fourier space corresponds to the narrow direction in real space.
I also normalized the profile to have unit flux.

To expand in size, we use
\begin{equation}
  \Delta = 1 - \left(\frac{r_1}{r_0}\right)^2,
\end{equation}
or
\begin{equation}
  r_0^2 = r_1^2/(1-\Delta),
\end{equation}
where $r_1$ is the size of the nearest precomputed profile.
Notice that the fraction above is the inverse of that used in the Spergel profile expansion.

For convenience we take $r_1 \rightarrow 1$ and expand:
\begin{align}
  \Sigma(\vec{r})
  &= \frac{L(\beta-1)\sqrt{1-\epsilon^2}(1-\Delta)}{\pi \{1+r^2(1-\Delta)[1-\epsilon\cos(2(\phi-\phi_0))]\}^\beta}
  \annot{substitute in $\Delta$} \\
  &= \frac{L(\beta-1)\sqrt{1-\epsilon^2}(1-\Delta)}{\pi \{1+r^2 - r^2[\Delta + (1-\Delta)\epsilon\cos(2(\phi-\phi_0))]\}^\beta}
  \annot{expand $(1-\epsilon ... )$} \\
  &= \frac{L(\beta-1) (1+r^2)^{-\beta}\sqrt{1-\epsilon^2}(1-\Delta)}{\pi \{1 - \frac{r^2}{1+r^2}\left[\Delta + (1-\Delta)\epsilon\cos(2(\phi-\phi_0))\right]\}^\beta}
  \annot{divide by $(1+r^2)^\beta$} \\
  &= \frac{L(\beta-1)\sqrt{1-\epsilon^2}(1-\Delta)}{\pi (1+r^2)^\beta}\sum_{j=0}^{\infty}\binom{\beta-1+j}{j}\left(\frac{r^2}{1+r^2}\right)^{j}[\Delta + (1-\Delta)\epsilon\cos(2(\phi-\phi_0))]^j
  \annot{binomial expansion of $\{\cdot\}^\beta$} \\
  &= \sum_{j=0}^{\infty}\frac{L (\beta-1)\sqrt{1-\epsilon^2}(1-\Delta) r^{2j}}{\pi (1+r^2)^{\beta+j}}\binom{\beta-1+j}{j}[\Delta + (1-\Delta)\epsilon\cos(2(\phi-\phi_0))]^j
  \annot{regroup terms} \\
  &= \sum_{j=0}^{\infty}\sum_{m=0}^{j}\frac{L (\beta-1) \sqrt{1-\epsilon^2} r^{2j}}{\pi (1+r^2)^{\beta+j}}\binom{\beta-1+j}{j}\binom{j}{m}\Delta^{j-m}(1-\Delta)^{m+1}\epsilon^m\cos^m(2(\phi-\phi_0))
  \annot{binomial expansion of $[\cdot]^j$ and combine $(1-\Delta)$s} \\
  &= \sum_{j=0}^{\infty}\sum_{m=0}^{j}\sum_{n=0}^{m}\frac{L (\beta-1) \sqrt{1-\epsilon^2} r^{2j}}{\pi (1+r^2)^{\beta+j}}\binom{\beta-1+j}{j}\binom{j}{m}\Delta^{j-m}(1-\Delta)^{m+1}\epsilon^m\exp(i 2 (2n-m) (\phi-\phi_0))
  \annot{expand $\cos^m(...))$} \\
  &= \sum_{j=0}^{\infty}\sum_{m=0}^{j}\sum_{n=0}^{m} a_{jmn}\mu_{jmn}(\vec{r}),
\end{align}
where
\begin{equation}
  a_{jmn}(L, \Delta, \epsilon, \phi_0) = \frac{L}{2^m} \sqrt{1-\epsilon^2} \Delta^{j-m}(1-\Delta)^{m+1}\epsilon^m\binom{j}{m}\binom{m}{n}\exp(-i 2 (2n-m) \phi_0)
\end{equation}
and
\begin{equation}
  \mu_{jmn}(\vec{r}) = \frac{(\beta-1)r^{2j}}{\pi (1+r^2)^{\beta+j}}\binom{\beta-1+j}{j}\exp(i 2 (2n-m) \phi).
\end{equation}

We have arranged factors that could have occured in either $a_{jmn}$ or $\mu_{jmn}(\vec{r})$ in such a way that $\mu_{jmn}(\vec{r})$ only depends on $j$ and $q = 2n-m$ (and hence not on both $m$ and $n$ separately).
$q$ can take on values between $-j$ and $j$.
For given $j$ and $q$, the range of possible $m$ is $|q| < m < j$.
For given $q$ and $m$ the value of $n$ is determined from the equation above as $n=(m+q)/2$.
This allows us to rewrite the triple sum as
\begin{equation}
  \label{eqn:sum_jq}
  \Sigma(\vec{r}) = \sum_{j=0}^{\infty}\sum_{q=-j}^{j} a_{jq}\mu_{jq}(\vec{r}),
\end{equation}
where
\begin{equation}
  \label{eqn:ajq}
  a_{jq}(L, \Delta, \epsilon, \phi_0) = \sum_{\substack{m=|q|\\m+q\,\mathrm{even}}}^{j}\frac{L}{2^m} \sqrt{1-\epsilon^2}\Delta^{j-m}(1-\Delta)^{m+1}\epsilon^m\binom{j}{m}\binom{m}{\frac{m+q}{2}}\exp(-i 2 q \phi_0)
\end{equation}
and
\begin{equation}
  \label{eqn:mujq}
  \mu_{jq}(\vec{r}) = \frac{(\beta-1)r^{2j}}{\pi(1+r^2)^{\beta+j}}\binom{\beta-1+j}{j}\exp(i 2 q \phi).
\end{equation}

The coefficients $a_{jq}$ and basis functions $\mu_{jq}(\vec{r})$ are complex, but we know the surface brightness profile $\Sigma(\vec{r})$ is real (see Eq. \ref{eqn:SBdef}).
There must be a good amount of cancellation in the imaginary parts of the terms of the series.
Inspecting Eqs. \ref{eqn:ajq} and \ref{eqn:mujq}, we see that
\begin{equation}
  a_{jq}\mu_{jq}(\vec{r}) = [a_{j(-q)}\mu_{j(-q)}(\vec{r})]^*,
\end{equation}
which implies
\begin{equation}
  \Re{[a_{jq}\mu_{jq}(\vec{r}) + a_{j(-q)}\mu_{j(-q)}(\vec{r})]} = 2\Re{[a_{jq}\mu_{jq}(\vec{r})]}
  = 2 (\Re{[a_{jq}]}\Re{[\mu_{jq}(\vec{r})]} - \Im{[a_{jq}]}\Im{[\mu_{jq}(\vec{r})]})
\end{equation}
and
\begin{equation}
  \Im{[a_{jq}\mu_{jq}(\vec{r}) + a_{j(-q)}\mu_{j(-q)}(\vec{r})]} = 0.
\end{equation}

We can use these relations to define $a^\prime_{jq}$ and $\mu^\prime_{jq}(\vec{r})$
\[
a^\prime_{jq} =
\begin{cases}
  2 \Re[a_{jq}], & q > 0, \\
  \Re[a_{jq}], & q = 0, \\
  -2 \Im[a_{jq}], & q < 0,
\end{cases}
\]
\[
\mu^\prime_{jq}(\vec{r}) =
\begin{cases}
  \Re[\mu_{jq}(\vec{r})], & q \ge 0, \\
  \Im[\mu_{jq}(\vec{r})], & q < 0,
\end{cases}
\]
such that
\begin{equation}
  \label{eqn:sum_jqprime}
  \Sigma(\vec{r}) = \sum_{j=0}^{\infty}\sum_{q=-j}^{j} a^\prime_{jq}\mu^\prime_{jq}(\vec{r}),
\end{equation}
where all the basis functions and coefficients are real (which is good, because GalSim doesn't include the ability to generate complex-valued images.)
Notice that we have been careful to not double count the terms $q$ and $-q$ when $q=0$.

For a given series order $j_\mathrm{max}$, there are $(j_\mathrm{max}+1)^2$ terms.
\end{document}
